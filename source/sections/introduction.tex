\documentclass[../main.tex]{subfiles}

\begin{document}
Dịch bệnh COVID-19 do virus corona chủng mới gây ra được phát hiện lần đầu tiên ở thành phố Vũ Hán, Trung Quốc từ tháng 12 năm 2019 \cite{WHOcoronavirus}. Dịch bệnh sau đó đã lây ra toàn Trung Quốc và khắp nới trên thế giới \cite{WHOoutbreaknews, WHOsituation}. Một người bình thường có thể nhiễm bệnh nếu có tiếp xúc gần với người bị bệnh. Các dấu hiệu và triệu chứng được báo cáo gồm sốt, mệt mỏi, ho khan, khó thở và suy hô hấp. Hấu hết các bệnh nhân đều có các triệu chứng nhẹ và có tiên lượng tốt. Các ca bệnh dẫn tới tử vong thường là người đã có độ tuổi cao và có các bệnh lý nền như tim mạch và đái tháo đường \cite{WHOcoronavirus}. Cho đến ngày 15 tháng 2 năm 2020 đã có 68500 trường hợp nhiễm bệnh và 1596 ca tử vong được báo cáo ở Trung Quốc \cite{NHCChina}.

Chính phủ Trung Quốc đã sử dụng nhiều biện pháp cách ly và giãn cách xã hội đã giảm sự lan truyền của dịch bệnh trong cộng đồng. Điển hình như là tìm kiếm, cách ly và theo dõi  14 ngày những người có tiếp xúc trực tiếp với người bệnh. Từ ngày 23 tháng 1 năm 2020, những nơi dịch bệnh bùng phát như Vũ Hán được phong tỏa hoàn toàn. Các trường học và doanh nghiệp được tạm ngừng để phòng tránh dịch bệnh lây lan. Người dân ở một vài quốc gia không được cấp VISA để nhập cảnh vào Trung Quốc. Nhiều chuyến bay trong và ngoài nước cũng bị hủy do dịch bệnh hoành hành.

Ở thập kỷ trước, thế giới cũng đã trải qua một số đại dịch như dịch SARS năm 2003, dịch H1N1 năm 2009, dịch H7N9 năm 2009. Do đó mà đã tạo ra nhu cầu về các dự báo sớm và sự nguy hiểm của các dịch bệnh. Các mô hình toán học nhờ đó đã được phát triển để nghiên cứu và phân tích sự lan truyền của dịch bệnh và đề ra các giải pháp phòng chống dịch bệnh. Ví dụ như Zhou và Ma \cite{sars1} đã tạo ra một mô hình toán học rời rạc để nghiên cứu sự lan truyền của dịch bệnh SARS. Kết quả mô phỏng của họ cho thấy việc cách ly sớm là yếu tố quan trọng trong việc ngăn chặn dịch bệnh. Chowell \cite{sars2} và Lekone \cite{seir} đã tạo ra mô hình tất định sử dụng các phương trình vi phân, và mô hình bất định SEIR để nghiên cứu sự lan truyền của dịch bệnh và sự hiệu quả của các biện pháp phòng dịch. Mô hình của họ được kiểm nghiệm dựa trên dữ liệu của đợt bùng phát dịch bệnh Ebola ở Cộng hòa Công Gô năm 1995.

Trong báo cáo này, chúng tôi sẽ sử dụng mô hình ngẫu nhiên rời rạc để dự báo sự lan truyền của dịch COVID-19 ở Trung Quốc. Mô hình sẽ mô phỏng sự lan truyền của dịch bệnh và sự ảnh hưởng của các biện pháp ngăn chặn dịch bệnh của chính phủ Trung Quốc. Các dữ liệu báo cáo về dịch bệnh sẽ được dùng để ước lượng tham số của mô hình. Từ đó, mô hình sẽ được áp dụng để dự báo sự lan truyền của dịch bệnh trong khoảng thời gian kế tiếp. Ngoài ra, mô hình cũng được sử dụng để tính toán thời gian hợp lý để trở lại cuộc sống bình thường của người dân Trung Quốc.
\end{document}