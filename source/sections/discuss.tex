\documentclass[../main.tex]{subfiles}

\begin{document}
Bằng việc xây dựng một mô hình ngẫu nhiên rời rạc để nghiên cứu sự lây truyền của dịch bệnh COVID-19, ta đã có thể quan sát được tính bất định trong sự lây lan của dịch bệnh. Trong mô hình này, chúng tôi đã giả định rằng tỷ lệ tiếp xúc với người bị nhiễm bệnh sẽ giảm theo dần theo thời gian nhờ vào việc áp dụng các biện pháp phòng dịch hiệu quả. Dữ liệu để phục vụ cho việc ước lượng tham số của mô hình được chúng tôi thu thập từ ngày 11 tháng 1 đến ngày 13 tháng 2 năm 2020. Ngoài ra, để ước lượng được tham số của mô hình, chúng tôi đã sử dụng phương pháp Markov Chain Monte Carlo (MCMC).

Giá trị ban đầu của tỷ lệ tiếp xúc khi chưa được áp dụng các biện pháp phòng dịch là $34.02$ và sau đó giảm dần theo thời gian xuống còn $0.93$. Điều này cho thấy các biện pháp phòng chống dịch bệnh là hiệu quả. Hệ số lây nhiễm hiệu quả được cho là rất cao trong những giai đoạn đầu của dịch, tượng trưng cho việc dịch bệnh COVID-19 có thể bùng phát một cách nhanh chóng. Tuy nhiên, nó cũng giảm dần theo thời gian xuống dưới 1, tức là sự bùng phát của dịch bệnh sẽ được dập tắt.

Sự mô phỏng của mô hình cho thấy nó không không phù hợp với các số liệu đã được báo cáo trong giai đoạn đầu của dịch bệnh, nhưng ở nhưng giai đoạn sau nó mô phỏng khá tốt với các số liệu mô phỏng phù hợp với các số liệu ở thực tế. Một minh chứng thực tế là dịch bệnh COVID-19 ở Trung Quốc đã được kiểm soát và số ca nhiễm còn lại đã giảm tiệm cần về không trong những ngày cuối tháng 4 và đầu tháng 5 năm 2020, điều mà đã được mô mình dự báo trước đó. Để xem xét thời điểm thích hợp để người dân có thể trở lại cuộc sống bình thường, chúng tôi đã xem xét bốn trường hợp khác nhau. Theo đó, chúng tôi giả sử tỷ lệ tiếp xúc $c(t)=3$ nếu như không có các biện pháp phù hợp để bảo vệ người dân khỏi dịch bệnh khi trở lại cuộc sống thường nhật. Trong trường hợp này, mô phỏng cho thấy rằng dịch bệnh sẽ có một đợt bùng phát thứ hai nếu người dân trở lại cuộc sống bình thường vào ngày 1 tháng 3 năm 2020. Với cùng tỷ lệ tiếp xúc đó, tình hinh dịch bệnh sẽ tốt hơn nếu người dân trở lại cuộc sống bình thường vào ngày 20 tháng 3 năm 2020.

Tóm lại, mô phỏng của chúng tôi cho thấy rằng tỷ lệ tiếp xúc là yếu tố chính trong việc kiểm soát dịch bệnh COVID-19. Việc không kiểm soát được tỷ lệ tiếp xúc như ở Mỹ và các nước châu Âu đã làm cho dịch bệnh bùng phát mạnh mẽ và chưa có điểm dừng cho tới thời điểm hiện tại. Ngoài ra, các biện pháp bảo vệ cá nhân như đeo khẩu trang và hạn chế di chuyển khi có dịch bệnh là một cách hiệu quả để giảm nguy cơ lây nhiễm cộng đồng. Thực tế cho thầy rằng có nhiều khả năng gây ra các đợt bùng phát tiếp theo mà mô hình của chúng tôi chưa thể dự báo trước được, ví dụ như các biến thể mới của virus corona. Vì vậy mà mọi người nên cảnh giác trước sự lây lan của dịch bệnh này.
\end{document}